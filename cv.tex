\documentclass[10pt,a4paper,sans]{moderncv}
\moderncvstyle{casual}
\moderncvcolor{black}
%\usepackage[inner=4cm,outer=2cm]{geometry} %left=4cm,right=2cm would be equivalent
\usepackage[scale=0.9]{geometry}
%\usepackage[margin=0.3in]{geometry}
\usepackage{multicol}
%\usepackage[pagebackref=false,colorlinks,linkcolor=blue,citecolor=magneta]{hyperref}
%\hypersetup{bookmarks,bookmarksopen,bookmarksdepth=2}
\firstname{Anastasiia}
\familyname{Tsarkova}

\email{ailyshnikova@gmail.com}
%\photo[50pt][0.4pt]{photo5}

\begin{document}
\makecvtitle

\section{Education}
\cventry{September 2014 -- June 2018}{BS in Applied Mathematics and Information Science}{National Research University Higher School of Economics, Moscow, Russia}{GPA -- 7.73 / 10 (\color{blue} \href{https://drive.google.com/file/d/1jo0wNmUpG3tNjdY4JZqNwLyAL8uNWQpm/view}{transcript}\color{black})}{}{}

\section{Projects}

All projects are stored in git repository: \url{https://github.com/ilyshnikova/}
\newline

\section{Work experience}
\cventry{July 2018 - now}{Software Engineer}{Google Switzerland}{Algorithmic Reconciliation for Knowledge Engine}{}{
	\begin{itemize}
		\item{Deprecated old Musical Recording Clustering pipeline and migrated it onto the horizontal infrastructure. Saved eng cost and maintenance cost (approx 1 engineer). Upon migration faced 70\% quality drop, which I did an analysis for and came up with a proposal to bring the quality back to what it was. I also coordinated the whole migration work end to end, including finalizing rollout procedures.}
		\item{Migrated Google Podcasts Clustering System to the horizontal infrastructure. Quality analysis was required in order to guarantee quality parity. Interacted with 2 teams to achieve a consistent plan to finalize that migration.}
		\item{Worked on a new generation of Shopping Product Clustering pipeline, which was built from scratch. The pipeline processes billions of Shopping offers and catalogs, which required careful considerations with regards to resource usage.}
		\item{Became primary go-to person for Finance and Geo verticals of Knowledge Engine Reconciliation.}
		\item{Significantly improved freshness of Geo reconciliation from yearly to daily.}
		\item{Frequently debugging release blockers and becoming go-to person for analysis of such blockers.}
	\end{itemize}
}
\cventry{September 2017 -- March 2018}{Software Engineer}{Yandex}{Advertising Services, Group of Product Development}{}{
	Continued to work in the team, which develops real-time advertisement service.
}
\cventry{June 2017 -- August 2017}{Business Intern}{Google}{Trust \& Safety, Gmail Abuse Group}{}{
  The main tasks of internship: to facilitate the infrastructure, automate the calculation of various statistics with map-reduce job, come up with metrics around precision and recall and create dashboards to monitor abuse.
	\newline
	Technologies: C++, Python, SQL
}
\cventry{August 2016 -- June 2017}{Software Engineer}{Yandex}{Advertising Services, Real-time Technologies Development Group}{}{
	Working in a group developed high-load real-time service, which processes a large number of advertisement requests.
	\newline
	Responsibilities:
	    \begin{itemize}
		    \item{Developing and maintaining product logic and infrastructural components}
		    \item{Debugging and supporting the service}
	    \end{itemize}
	Tasks of real-time advertisement engine I worked on required algorithmic knowledge as well as good time management. Also, it was a useful experience of working in a rather large team and reading a lot of business-oriented code of a huge system.
  	\newline
	Technologies: C++, Python, Perl, SQL, gdb
}
\cventry{April 2016 -- July 2016}{Software Engineer Intern}{Yandex}{Geo-Informational Services, Real-time Services Group}{}{
	Implemented testing framework for the geo-service engine, which emulates loaded traffic on any given geographic graph. Took part in some of the infrastructural tasks of real-time search service.
}


\section{Courseworks}
\cventry{2017--now}{Improvement of Analytical DBMS \color{blue} \href{https://clickhouse.yandex/}{ClickHouse}\color{black}, NRU HSE}{NRU HSE, Diploma Project}{}{}{
ClickHouse is an open source column-oriented database management system capable of real-time generation of analytical data reports using SQL queries. My role for this project was to solve the following problems:
		\begin{itemize}
			\item To outline the possibility to force usage of ClickHouse's MergeTree engine table primary key in analyzing the query conditional, containing 'in'-clause with multiple fields. This feature was implemented and is already in the stable branch.
			\item To implement conditional computations for logical expressions such as 'and'/'or' operators and build a logic to identify in which cases conditional computations work faster. This part is in development state.
		\end{itemize}
		My pull requests can be found \color{blue} \href{https://github.com/yandex/ClickHouse/pulls?q=is\%3Apr+author\%3Ailyshnikova}{here}\color{black}
}

\cventry{2016--2017}{Statistical Analysis of Financial Data by Means of Machine Learning,}{NRU HSE, Team Project}{}{}{
		\begin{itemize}
			\item System of indicative statistical and qualitative analysis, which forecasts efficient investments
			\item Implemented library, which provides user with easy way to download and analyze financial data from different sources
			\item Includes human interface to facilitate work for users without programming skills
		\end{itemize}
}
\cventry{2015--2016}{Graphs Analyser Utility}{NRU HSE}{}{}{
		\begin{itemize}
			\item Server-based program, which accepts time series points and detects anomalies on data series.
			\item Works online with complexity $\underline{O}(1)$ on point submission.
			\item Includes human html interface for controlling algorithms, which work on time series.
			\item Source code: \url{https://github.com/ilyshnikova/graph-analyzer}
		\end{itemize}
}


\section{Skills}

\cventry{Programming}{C++, Python, Perl, JavaScript (basic), html, SQL}{}{}{}{}
\cventry{ML tools}{Pandas, NumPy, SciPy, scikit-learn, matplotlib, Apache Spark, Keras, Vowpal Wabbit, word2vec}{}{}{}{}
\cventry{Other tools}{linux, mac os, git, svn, deb packages, gdb, pdb, vim, flask, nginx}{}{}{}{}
\cventry{Languages}{Russian,English}{}{}{}{}

\end{document}

